\documentclass[a4paper,12pt,twocolumn]{article}
\usepackage[latin1]{inputenc}
\usepackage[english]{babel}
\usepackage{amsmath}
\usepackage{amsfonts}
\usepackage{amssymb}
\usepackage{mathtools}

% Set top and bottom to 1in too.
\usepackage[left=1in,right=1in,top=0.8in,bottom=0.8in]{geometry}

\usepackage{titling}
\usepackage{nomencl}
\usepackage{siunitx}
\usepackage[style=ieee,backend=bibtex]{biblatex}
\usepackage[font={small}]{caption}

\usepackage{graphicx}
\usepackage{color}
\usepackage[table]{xcolor}

\usepackage{booktabs}
\usepackage{threeparttable}
\usepackage{multirow}
\usepackage{fancyhdr}
\usepackage{float}

\usepackage{varioref}
\usepackage{xspace}
\usepackage[activate={true,nocompatibility},final,tracking=true,kerning=true,spacing=nonfrench,factor=1100,stretch=10,shrink=10]{microtype}
% activate={true,nocompatibility} - activate protrusion and expansion final -
% enable microtype; use "draft" to disable tracking=true, kerning=true,
% spacing=true - activate these techniques factor=1100 - add 10% to the
% protrusion amount (default is 1000) stretch=10, shrink=10 - reduce
% stretchability/shrinkability (default is 20/20) Reduce tracking around small
% caps to 40%
\SetTracking{encoding={*}, shape=sc}{40}

% Document info.
\author{Z0966990}
\title{Numerical Methods in \textsc{Matlab}}
\date{\today}

% Path to images.
\graphicspath{{img/}}

% Setup nomenclature.
\newlength{\nomtitlesep}
\setlength{\nomtitlesep}{\nomitemsep}
\setlength{\nomitemsep}{-0\parsep}
\renewcommand\nomgroup[1]{
    \ifthenelse{\equal{#1}{A}}{
        \itemsep\nomtitlesep
        \item[\textbf{Acronyms}]
        \itemsep\nomitemsep}{  
}}
\newcommand{\nomindex}[1]{
    \hspace*{\fill}
    \makebox[5em][l]{#1}
}
\makenomenclature

% Setup bibiliography.

% Header and footer.
\pagestyle{fancy}
\fancyhf{}
\lhead{\thetitle}
\rhead{\theauthor}
\cfoot{\thepage}
\renewcommand{\headrulewidth}{0pt}
\renewcommand{\footrulewidth}{0pt}

% Macros.

\begin{document}
    
% Title page.
\begin{titlepage}
    \centering
    \vspace*{\fill}
    \includegraphics[width=0.5\textwidth]{Durham.png}\\
    \vspace*{\fill}
    \LARGE\thetitle\\
    \large\theauthor\\
    \large L2 Engineering Mathematics\\
    \large\thedate\\
    \vspace*{\fill}
\end{titlepage}

%Main matter.
% \renewcommand{\abstractname}{\large Abstract}
% \twocolumn[
% \begin{@twocolumnfalse}
%     \begin{abstract}
%     \end{abstract}
% \end{@twocolumnfalse}
% \vspace{\parsep}
% ]

% Acronyms

\printnomenclature

\section{Taylor Series Method}

The $p$\textsuperscript{th} order Taylor's series method for ODEs finds
numerical approximations to equations of the form $y'(t) = f(t,y)$. In $n$ steps
it approximates the solution $y_n$ at some time $T$ given an initial condition:
\begin{footnotesize}
    \vspace{-0.5\baselineskip}
    \begin{equation} \label{eq:taylor-method}
        \begin{aligned}
            t_0 &= 0 &
            y_0 &= y(0) \\
            t_{i+1} &= t_i + h &
            y_{i+1} &= y_i + \sum_{k=1}^{p}\frac{h^k}{k!}f^{(k-1)}(t_i,y_i) \\
            t_n &= T &
            y_n &\approx y(T) 
        \end{aligned}
    \end{equation}
    \vspace{-1.5\baselineskip}
\end{footnotesize}

\hspace{-\parindent}where $h = T/n$ and is the stepsize.

The use of successive approximations results in an error composed of the
accumulation error and truncation error at each step. At time $T$, this is
the final global error:
\begin{footnotesize}
    \vspace{-0.5\baselineskip}
    \begin{equation} \label{eq:error}
        \epsilon_n = |y_n - y(T)| = O(h^p) = O(n^{-p})
    \end{equation}
    \vspace{-1.5\baselineskip}
\end{footnotesize}

A \textsc{Matlab} implementation of \eqref{eq:taylor-method} with order 4 was
applied to three ODEs of the form $y'(t) = f(t,y)$. These were defined as
follows:
\begin{footnotesize}
    \vspace{-0.5\baselineskip}
    \begin{align}
        \tag{a}\label{ode:a}
        y'(t) &= \sin(t) & y(0) &= 0 & 0 &\leq t \leq \pi  \\
        \tag{b}\label{ode:b}
        y'(t) &= 5y(t)   & y(0) &= 1 & 0 &\leq t \leq 0.1 \\
        \tag{c}\label{ode:c}
        y'(t) &= 2       & y(0) &= 0 & 0 &\leq t \leq 1
    \end{align}
    \vspace{-1.5\baselineskip}
\end{footnotesize}

In order to apply the fourth order Taylor's series method, the 
1\textsuperscript{st}--3\textsuperscript{rd} derivatives of $f(t, y)$ had
to be found:
\begin{table}[H]
    \centering
    \footnotesize
    \vspace{-\baselineskip}
    \begin{tabular}{rrrr}
        & ODE~\eqref{ode:a} & ODE~\eqref{ode:b} & ODE~\eqref{ode:c} \\
        \cmidrule(lr){2-2}\cmidrule(lr){3-3}\cmidrule(lr){4-4}
        $f(t,y)$\tnote{*} & $\sin(t)$  & $5y$   & 2 \\
        $f'(t,y)$         & $\cos(t)$  & $25y$  & 0 \\
        $f''(t,y)$        & $-\sin(t)$ & $125y$ & 0 \\
        $f'''(t,y)$       & $-\cos(t)$ & $625y$ & 0 \\
    \end{tabular}
\vspace{-\baselineskip}
\end{table}

\vspace{-\baselineskip}
\subsection{Results}

The results of the fourth order Taylor series method were compared to the 
results of Taylor's second and Euler's method---which is a first order 
Taylor's series method. Table~\ref{table:ode} lists $y_n$ for different numbers 
of iterations $n$.
\begin{table}[h]
    \centering
    \footnotesize
    \vspace{-0.5\baselineskip}
    \caption{ODE Verification Simulations}
    \label{table:ode}
    \vspace{-0.5\baselineskip}
    \begin{threeparttable}
        \begin{tabular}{
            @{}
            l
            S[table-format=2]
            S[table-format=1.2]
            S[table-format=1.2]
            S[table-format=1.2]
            @{}
        }
            \toprule
            & $n$
            & {\bf Euler\tnote{*}}
            & {\bf Taylor 2\textsuperscript{nd}\tnote{*}}
            & {\bf Taylor 4\textsuperscript{th}\tnote{*}} \\
            \midrule
            \multirow{4}{*}{\rotatebox[origin=c]{90}{
                \shortstack{ODE~\eqref{ode:a}\\at $t=\mathrm{\pi}$}}}
            & 10 & 1.99 & 2.03 & 2.00 \\
            & 20 & 2.00 & 2.01 & 2.00 \\
            & 40 & 2.00 & 2.00 & 2.00 \\
            & 80 & 2.00 & 2.00 & 2.00 \\
            \midrule
            \multirow{4}{*}{\rotatebox[origin=c]{90}{
                \shortstack{ODE~\eqref{ode:b}\\at $t=0.1$}}}
            & 10 & 1.63 & 1.65 & 1.65 \\
            & 20 & 1.64 & 1.65 & 1.65 \\
            & 40 & 1.64 & 1.65 & 1.65 \\
            & 80 & 1.65 & 1.65 & 1.65 \\
            \midrule
            \multirow{4}{*}{\rotatebox[origin=c]{90}{
                \shortstack{ODE~\eqref{ode:c}\\at $t=1$}}}
            & 10 & 2.00 & 2.00 & 2.00 \\
            & 20 & 2.00 & 2.00 & 2.00 \\
            & 40 & 2.00 & 2.00 & 2.00 \\
            & 80 & 2.00 & 2.00 & 2.00 \\
            \bottomrule
        \end{tabular}
        \begin{tablenotes}
            \item[*] Accurate to 3 significant figures.
        \end{tablenotes}
    \end{threeparttable}
    \vspace{-\baselineskip}
\end{table}

\vspace{-\baselineskip}
ODE~\eqref{ode:c} had no non-zero derivatives in $f(t,y)$, so Euler's method
contained sufficiently many terms to find $y_n$ when $T=1$; higher order
methods behaved as Euler's method. For ODEs \eqref{ode:b} and \eqref{ode:c},
many more iterations were required for the lower order methods to converge on a
value for $y_n$.

For each ODE, analytical solutions were provided:
\begin{footnotesize}
    \vspace{-0.5\baselineskip}
    \begin{align}
        y(t) &= -\cos(t) + 1    \tag{\sc\ref{ode:a}}\label{soln:a} \\ 
        y(t) &= \mathrm{e}^{5t} \tag{\sc\ref{ode:b}}\label{soln:b} \\
        y(t) &= 2t              \tag{\sc\ref{ode:c}}\label{soln:c}
    \end{align}
    \vspace{-1.5\baselineskip}
\end{footnotesize}

\hspace{-\parindent}where \eqref{soln:a}--\eqref{soln:c} are the exact solutions
to ODEs \eqref{ode:a}--\eqref{ode:c}.

Using these solutions, the exact errors of the results in Table~\ref{table:ode}
can be found. These are detailed in Table~\ref{table:err}.
\begin{table}[h]
    \centering
    \footnotesize
    \vspace{-0.5\baselineskip}
    \caption{ODE Errors}
    \label{table:err}
    \vspace{-0.5\baselineskip}
    \begin{threeparttable}
        \begin{tabular}{
            @{}
            l
            S[table-format=2]
            S[table-format=1.2e+2]
            S[table-format=1.2e+2]
            S[table-format=1.2e+2]
            @{}
        }
            \toprule
            & {$n$}
            & {\bf\shortstack{Euler\\Error\tnote{*}}}
            & {\bf\shortstack{Taylor 2\textsuperscript{nd}\\Error\tnote{*}}}
            & {\bf\shortstack{Taylor 4\textsuperscript{th}\\Error\tnote{*}}} \\
            \midrule
            \multirow{4}{*}{\rotatebox[origin=c]{90}{
                \shortstack{ODE~\eqref{ode:a}\\at $t=\mathrm{\pi}$}}}
            & 10 & 1.65e-02 & 3.29e-02 & 1.62e-04 \\
            & 20 & 4.11e-03 & 8.22e-03 & 1.01e-05 \\
            & 40 & 1.03e-03 & 2.06e-03 & 6.34e-07 \\
            & 80 & 2.57e-04 & 5.14e-04 & 3.96e-08 \\
            \midrule
            \multirow{4}{*}{\rotatebox[origin=c]{90}{
                \shortstack{ODE~\eqref{ode:b}\\at $t=0.1$}}}
            & 10 & 1.98e-02 & 3.31e-04 & 4.12e-08 \\ 
            & 20 & 1.01e-02 & 8.43e-05 & 2.63e-09 \\ 
            & 40 & 5.10e-03 & 2.13e-05 & 1.66e-10 \\ 
            & 80 & 2.56e-03 & 5.34e-06 & 1.04e-11 \\ 
            \midrule
            \multirow{4}{*}{\rotatebox[origin=c]{90}{
                \shortstack{ODE~\eqref{ode:c}\\at $t=1$}}}
            & 10 & 2.22e-16 & 2.22e-16 & 2.22e-16 \\ 
            & 20 & 4.44e-16 & 4.44e-16 & 4.44e-16 \\ 
            & 40 & 8.88e-16 & 8.88e-16 & 8.88e-16 \\ 
            & 80 & 3.11e-15 & 3.11e-15 & 3.11e-15 \\ 
            \bottomrule
        \end{tabular}
        \begin{tablenotes}
            \item[*] Accurate to 3 significant figures.
        \end{tablenotes}
    \end{threeparttable}
    \vspace{-\baselineskip}
\end{table}

ODEs \eqref{ode:a} and \eqref{ode:b} demonstrate how increasing the number of
iterations reduces the final global error, as suggested in \eqref{eq:error}.
Equation~\eqref{eq:error} also suggests increasing the order of the Taylor's
series method reduces the final global error. Particularly for
ODE~\eqref{ode:b}, doubling the order of the method halved the magnitude of the
error.

However, it was established for ODE~\eqref{ode:c} that higher order methods
behaved as Euler's method, so the order of the method has no effect on accuracy.
Contray to \eqref{eq:error} the error increased with the number of iterations.
This can be explained by the error in encoding the fraction $h$---for example,
when $n=10$, \textsc{Matlab}'s binary64 format cannot encode every bit of the
recurring binary fraction $h=0.1=0.0\dot{0}01\dot{1}_2$. The termination of the
recurring fractions is called round off error, and is present for all chosen
values of $n$. As the number of iterations $n$ increase, more round off error
accumulates.

\vspace{-\baselineskip}
\section{Finite Difference\\Method}

In physics and engineering, the convection--diffusion--reaction equation
describes how species mass concentration $u$ varies in space $\mathbf{x}$ with
time $t$. The equation considers the effects of convection
$\mathbf{v}\cdot\nabla u$, wherein the mass has a net velocity $\mathbf{v}$;
diffusion $c^2\nabla^2u$, where random motion will cause the mass to disperse
over time; and reaction $du$, which acts as a sink or source for the total mass
in the system to change over time---often present when chemical reactions are
taking place.

The equation can also be applied to heat transfer, where $u$ represents heat.

In 1D, the convection--diffusion--reaction equation is defined as follows:
\begin{footnotesize}
    \vspace{-0.5\baselineskip}
    \begin{align}
        \label{eq:pde}
        \begin{split}
            \forall x,t &\in (a, b) \times (0, T] \\
            u_{t}(x,t) &= c^2u_{xx}(x,t) + vu_x(x,t) + du(x,t)
        \end{split} \\[0.2\baselineskip]
        \nonumber
        \forall t &\in (0, T] \left\{
        \begin{array}{l}
            u(a,t) = 0 \\
            u(b,t) = 0
        \end{array}
        \right.\\[0.2\baselineskip]
        \nonumber
        \forall x &\in [a, b]\hspace{1.8em}u(x,0) = f(x)
    \end{align}
    \vspace{-1.5\baselineskip}
\end{footnotesize}

One possible FDM schema was devised to numerically solve \eqref{eq:pde}, which
is second order in space with interval $h$ and first order in time with timestep
$k$. The schema was implemented in \textsc{Matlab} as follows:
\begin{footnotesize}
    \vspace{-0.5\baselineskip}
    \begin{multline} \label{eq:schema}
        u^i_{j+1} = u^i_j + dku^i_j + \frac{vk}{2h}(u^{i+1}_j - u^{i-1}_j) \\
            + \frac{c^2k}{h^2}(u^{i+1}_j - 2u^i_j + u^{i-1}_j)
    \end{multline}
    \vspace{-1.5\baselineskip}
\end{footnotesize}

Errors in FDM schema can grow or be damped out at each timestep, leading to a
stable and accurate, or unstable solutions. Von Neumann stability analysis 
assumes errors are amplified by a constant factor. An expression for
amplification factor can be determined by expressing the errors as a Fourier 
series in space. Stability necessitates the amplification factor is less than 
unity for all frequencies resolable in the grid.

Applying von Neumann stability analysis to \eqref{eq:schema} yielded the 
following stability criterion, and expression for the amplification factor:
\begin{footnotesize}
    \vspace{-0.5\baselineskip}
    \begin{gather}
        \label{eq:stability}
        G = \left| 1 + R - 2S \pm \sqrt{(2S)^2 + C^2} \right|
            \leq 1 + \epsilon k \\
        \nonumber
        \begin{aligned}
            R &= dk & C &= \frac{vk}{h} & S &= \frac{c^2k}{h^2}
        \end{aligned}
    \end{gather}
    \vspace{-1.5\baselineskip}
\end{footnotesize}

\hspace{-\parindent}where $R$, $C$ and $S$ are dimensionless quantities: the
reaction, Courant (convection) and diffusion numbers. $\epsilon$ is a
sufficently small constant which in the following analysis has been assumed to 
be smaller than \num{800e-3}.

The \textsc{Matlab} model and stability criterion were tested using the
convection--diffusion--reaction equation with the following coefficients:
\begin{footnotesize}
    \vspace{-0.5\baselineskip}
    \begin{align*}
        c &= 1    & a &= 0   & m,n  =&\;\{10,100,200,400,800\} \\
        v &= 0.5  & b &= 1   &       &\;\times \{4,16,32\} \\
        d &= 0.02 & T &= 0.2 & f(x) =&\;4x - 4x^2
    \end{align*}
    \vspace{-1.5\baselineskip}
\end{footnotesize}

With these coefficients fixed, the amplification factor in \eqref{eq:stability}
varies as follows:
\begin{footnotesize}
    \vspace{-0.5\baselineskip}
    \begin{equation} \label{eq:gain}
        G = O(k/h^2) = O(n^2/m)
    \end{equation}
    \vspace{-1.5\baselineskip}
\end{footnotesize}

\vspace{-\baselineskip}
\subsection{Results}

Table~\ref{table:pde} lists the solved value of \eqref{eq:pde} at the midpoint
of the interval at time $T$, for each combination of $n$ and $m$.

\begin{table}[h]
    \centering
    \footnotesize
    \vspace{-0.5\baselineskip}
    \caption{PDE Verification Simulations}
    \label{table:pde}
    \vspace{-0.5\baselineskip}
    \begin{threeparttable}
        \begin{tabular}{
            S[table-format=2]
            S[table-format=3]
            S[table-format=+2.2e2]
            S[table-format=2.2e+1]
        }
            \toprule
            {$n$} & {$m$} & {$u(0.5,0.2)$\tnote{*}} & {$G$\tnote{*}}\\
            \cmidrule(r){1-3}\cmidrule(l){4-4}
             4 &  10 &    0.129   &  1.00 \\\rowcolor{red!20}
            16 &  10 &   -1.13e7  & 19.5  \\\rowcolor{red!20}
            32 &  10 &  -87.3     & 80.9  \\
             4 & 100 &    0.155   &  1.00 \\\rowcolor{red!20}
            16 & 100 &    0.139   &  1.05 \\\rowcolor{red!20}
            32 & 100 &   -1.53e80 &  7.19 \\
             4 & 200 &    0.156   &  1.00 \\
            16 & 200 &    0.142   &  1.00 \\\rowcolor{red!20}
            32 & 200 &   -4.66e92 &  3.1  \\
             4 & 400 &    0.157   &  1.00 \\
            16 & 400 &    0.143   &  1.00 \\\rowcolor{red!20}
            32 & 400 &   -1.30e2  &  1.05 \\
             4 & 800 &    0.157   &  1.00 \\
            16 & 800 &    0.143   &  1.00 \\
            32 & 800 &    0.142   &  1.00 \\
            \bottomrule
        \end{tabular}
        \begin{tablenotes}
            \item\colorbox{red!20}{Unstable: unable to satisfy
                \eqref{eq:stability}.}
            \item[*] Accurate to 3 significant figures.
        \end{tablenotes}
    \end{threeparttable}
    \vspace{-\baselineskip}
\end{table}

The highlighted simulations with a gain greater than unity do not appear to
converge on similar values to those determined stable by \eqref{eq:stability}.
This occurs when there are too many intervals in space for a given number of
timesteps, as predicted by \eqref{eq:gain}.

% References.
\printbibliography

\end{document}